\section{Conclusions}

From a high-level viewpoint the parallel transducer approach by Ogden et
al.\cite{Ogden2013} and our approach are similar. In a first step, the input is
divided into equally sized chunks which are then processed independently. In the
case of Ogden et al., a state mapping is generated for each chunk by
speculatively executing the deterministic transducer in each possible starting
state. In our case, the tokenizer compresses each XML chunk into a tokenstream
chunk by mapping tag names to token IDs.

We could demonstrate strong scalability for the tokenizer, i.e., the throughput
increases almost linearly with the number of threads.

Both, the tokenstream and the state mappings, can be viewed as a transformed
representation of the original XML data which is suitable for a following,
necessarily sequential analysis: In the case of the parallel transducer, the
mappings are joined in a single-threaded process. In our case, the state
machines process the entire tokenstream.

It can be argued that our approach is less effective for a small number of
queries than the transducer approach, as each state machine needs to process the
entire tokenstream sequentially. However, we could demonstrate weak scalability
of our matcher: For up to 60 queries (the number of cores), the per-thread
throughput declines less than 20\% when compared to a single query. In the case
where a large number of queries must be executed, this is a clear advantage over
the parallel transducer approach, as for the latter the set of states
potentially grows exponentially with the size of the query list.

It is virtually impossible to make a direct comparison, as both, a
reimplementation of the parallel transducer approach was outside of the scope of
this project and the experimental results stem from different hardware setups.

In our experiments we reached a maximal throughput of a approximately 1.2 GBps
for the tokenizer running with 240 threads. As this is far below the maximal
memory bandwidth of 352 Gbps specified for the Xeon phi\cite{IntelXeon},
architectural optimization must be used to explore the limits of our system
design.

Furthermore, since we are dealing with an inherently memory bound problem, it is
likely that our system would profit from further compressing intermediate data.
For instance, as the structure of most XML files is highly repetitive, pattern
based compression schemes could be used to further compress the tokenstream.
Also, there exist efficient integer compression algorithms which could be
employed to compress the offset stream\cite{Lemire2014}.

The straight-forward, modular design of our system also allows for conceptual
enhancements. For instance, the tokenstream could be adapted to contain forward
pointers to sibling nodes that would allow a matcher to skip entire subtrees.
