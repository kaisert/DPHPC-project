\section{Conclusions}

The state mappings of the parallel pushdown transducer approach by Ogden et al
\cite{Ogden2013}  works on chunked XML data and produces state mappings from
all possible start states to their finishing states for each chunk individually. This 
introduces some overhead/additional computation although this enables strong
scaling during this step.


%WORK IN PROGRESS




% Original text:

%The state mappings of the parallel transducer approach can be considered to
%constitute a compressed representation of the original XML stream, similar to
%our tokenstream. Analogous to the generation of the state mapping, our approach
%scales linearly with the number of processors.
%
%The effectiveness of the parallel transducer stems from the fact that the state
%mappings can be efficiently traversed in the joining phase, which is necessarily
%sequential. Likewise, we try to reduce memory traffic in the matching phase.
%
%It is virtually impossible to make a direct comparison, as both a
%reimplementation of the parallel transducer approach was outside of the scope of
%this project and the experimental results stem from different hardware setups.
%It can be argued, though, that our approach is less effective for a small number
%of queries. However, we could demonstrate weak scalability for the query matcher
%allowing for up to 60 queries (the number of cores) to run in parallel.
%
%Furthermore, its straight-forward design lowers the cost of extensions to the
%system and also makes it easier to analyze. For example, supporting a larger
%subset of XPath merely introduces new tokens. Also, to further enhance
%performance, different techniques to compress the tokenstream could be employed.


