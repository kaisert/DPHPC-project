\section{Background}\label{sec:background}
In the following, we will give a brief overview over three general approaches
for parallel XML processing. For a more elaborate discussion we refer the read
to the background section of
%TODO: Citation.

\subsection{Parallel XML Querying}
In general, three approaches for XML query processing can be distinguished:

\begin{enumerate}
\item XML stream processing using automatons,
\item XML parsing and querying, and
\item XML-capable DBMS.
\end{enumerate}



%ive a short, self-contained summary of necessary
%ackground information. For example, assume you present an
%mplementation of sorting algorithms. You could organize into sorting
%efinition, algorithms considered, and asymptotic runtime statements. The goal of the
%ackground section is to make the paper self-contained for an audience
%s large as possible. As in every section
%ou start with a very brief overview of the section. Here it could be as follows: In this section 
%e formally define the sorting problem we consider and introduce the algorithms we use
%ncluding a cost analysis.

%mypar{Sorting}
%recisely define sorting problem you consider.

%mypar{Sorting algorithms}
%xplain the algorithm you use including their costs.

%s an aside, don't talk about "the complexity of the algorithm.'' It's incorrect,
%roblems have a complexity, not algorithms.

